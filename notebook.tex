
% Default to the notebook output style

    


% Inherit from the specified cell style.




    
\documentclass[11pt]{article}

    
    
    \usepackage[T1]{fontenc}
    % Nicer default font (+ math font) than Computer Modern for most use cases
    \usepackage{mathpazo}

    % Basic figure setup, for now with no caption control since it's done
    % automatically by Pandoc (which extracts ![](path) syntax from Markdown).
    \usepackage{graphicx}
    % We will generate all images so they have a width \maxwidth. This means
    % that they will get their normal width if they fit onto the page, but
    % are scaled down if they would overflow the margins.
    \makeatletter
    \def\maxwidth{\ifdim\Gin@nat@width>\linewidth\linewidth
    \else\Gin@nat@width\fi}
    \makeatother
    \let\Oldincludegraphics\includegraphics
    % Set max figure width to be 80% of text width, for now hardcoded.
    \renewcommand{\includegraphics}[1]{\Oldincludegraphics[width=.8\maxwidth]{#1}}
    % Ensure that by default, figures have no caption (until we provide a
    % proper Figure object with a Caption API and a way to capture that
    % in the conversion process - todo).
    \usepackage{caption}
    \DeclareCaptionLabelFormat{nolabel}{}
    \captionsetup{labelformat=nolabel}

    \usepackage{adjustbox} % Used to constrain images to a maximum size 
    \usepackage{xcolor} % Allow colors to be defined
    \usepackage{enumerate} % Needed for markdown enumerations to work
    \usepackage{geometry} % Used to adjust the document margins
    \usepackage{amsmath} % Equations
    \usepackage{amssymb} % Equations
    \usepackage{textcomp} % defines textquotesingle
    % Hack from http://tex.stackexchange.com/a/47451/13684:
    \AtBeginDocument{%
        \def\PYZsq{\textquotesingle}% Upright quotes in Pygmentized code
    }
    \usepackage{upquote} % Upright quotes for verbatim code
    \usepackage{eurosym} % defines \euro
    \usepackage[mathletters]{ucs} % Extended unicode (utf-8) support
    \usepackage[utf8x]{inputenc} % Allow utf-8 characters in the tex document
    \usepackage{fancyvrb} % verbatim replacement that allows latex
    \usepackage{grffile} % extends the file name processing of package graphics 
                         % to support a larger range 
    % The hyperref package gives us a pdf with properly built
    % internal navigation ('pdf bookmarks' for the table of contents,
    % internal cross-reference links, web links for URLs, etc.)
    \usepackage{hyperref}
    \usepackage{longtable} % longtable support required by pandoc >1.10
    \usepackage{booktabs}  % table support for pandoc > 1.12.2
    \usepackage[inline]{enumitem} % IRkernel/repr support (it uses the enumerate* environment)
    \usepackage[normalem]{ulem} % ulem is needed to support strikethroughs (\sout)
                                % normalem makes italics be italics, not underlines
    

    
    
    % Colors for the hyperref package
    \definecolor{urlcolor}{rgb}{0,.145,.698}
    \definecolor{linkcolor}{rgb}{.71,0.21,0.01}
    \definecolor{citecolor}{rgb}{.12,.54,.11}

    % ANSI colors
    \definecolor{ansi-black}{HTML}{3E424D}
    \definecolor{ansi-black-intense}{HTML}{282C36}
    \definecolor{ansi-red}{HTML}{E75C58}
    \definecolor{ansi-red-intense}{HTML}{B22B31}
    \definecolor{ansi-green}{HTML}{00A250}
    \definecolor{ansi-green-intense}{HTML}{007427}
    \definecolor{ansi-yellow}{HTML}{DDB62B}
    \definecolor{ansi-yellow-intense}{HTML}{B27D12}
    \definecolor{ansi-blue}{HTML}{208FFB}
    \definecolor{ansi-blue-intense}{HTML}{0065CA}
    \definecolor{ansi-magenta}{HTML}{D160C4}
    \definecolor{ansi-magenta-intense}{HTML}{A03196}
    \definecolor{ansi-cyan}{HTML}{60C6C8}
    \definecolor{ansi-cyan-intense}{HTML}{258F8F}
    \definecolor{ansi-white}{HTML}{C5C1B4}
    \definecolor{ansi-white-intense}{HTML}{A1A6B2}

    % commands and environments needed by pandoc snippets
    % extracted from the output of `pandoc -s`
    \providecommand{\tightlist}{%
      \setlength{\itemsep}{0pt}\setlength{\parskip}{0pt}}
    \DefineVerbatimEnvironment{Highlighting}{Verbatim}{commandchars=\\\{\}}
    % Add ',fontsize=\small' for more characters per line
    \newenvironment{Shaded}{}{}
    \newcommand{\KeywordTok}[1]{\textcolor[rgb]{0.00,0.44,0.13}{\textbf{{#1}}}}
    \newcommand{\DataTypeTok}[1]{\textcolor[rgb]{0.56,0.13,0.00}{{#1}}}
    \newcommand{\DecValTok}[1]{\textcolor[rgb]{0.25,0.63,0.44}{{#1}}}
    \newcommand{\BaseNTok}[1]{\textcolor[rgb]{0.25,0.63,0.44}{{#1}}}
    \newcommand{\FloatTok}[1]{\textcolor[rgb]{0.25,0.63,0.44}{{#1}}}
    \newcommand{\CharTok}[1]{\textcolor[rgb]{0.25,0.44,0.63}{{#1}}}
    \newcommand{\StringTok}[1]{\textcolor[rgb]{0.25,0.44,0.63}{{#1}}}
    \newcommand{\CommentTok}[1]{\textcolor[rgb]{0.38,0.63,0.69}{\textit{{#1}}}}
    \newcommand{\OtherTok}[1]{\textcolor[rgb]{0.00,0.44,0.13}{{#1}}}
    \newcommand{\AlertTok}[1]{\textcolor[rgb]{1.00,0.00,0.00}{\textbf{{#1}}}}
    \newcommand{\FunctionTok}[1]{\textcolor[rgb]{0.02,0.16,0.49}{{#1}}}
    \newcommand{\RegionMarkerTok}[1]{{#1}}
    \newcommand{\ErrorTok}[1]{\textcolor[rgb]{1.00,0.00,0.00}{\textbf{{#1}}}}
    \newcommand{\NormalTok}[1]{{#1}}
    
    % Additional commands for more recent versions of Pandoc
    \newcommand{\ConstantTok}[1]{\textcolor[rgb]{0.53,0.00,0.00}{{#1}}}
    \newcommand{\SpecialCharTok}[1]{\textcolor[rgb]{0.25,0.44,0.63}{{#1}}}
    \newcommand{\VerbatimStringTok}[1]{\textcolor[rgb]{0.25,0.44,0.63}{{#1}}}
    \newcommand{\SpecialStringTok}[1]{\textcolor[rgb]{0.73,0.40,0.53}{{#1}}}
    \newcommand{\ImportTok}[1]{{#1}}
    \newcommand{\DocumentationTok}[1]{\textcolor[rgb]{0.73,0.13,0.13}{\textit{{#1}}}}
    \newcommand{\AnnotationTok}[1]{\textcolor[rgb]{0.38,0.63,0.69}{\textbf{\textit{{#1}}}}}
    \newcommand{\CommentVarTok}[1]{\textcolor[rgb]{0.38,0.63,0.69}{\textbf{\textit{{#1}}}}}
    \newcommand{\VariableTok}[1]{\textcolor[rgb]{0.10,0.09,0.49}{{#1}}}
    \newcommand{\ControlFlowTok}[1]{\textcolor[rgb]{0.00,0.44,0.13}{\textbf{{#1}}}}
    \newcommand{\OperatorTok}[1]{\textcolor[rgb]{0.40,0.40,0.40}{{#1}}}
    \newcommand{\BuiltInTok}[1]{{#1}}
    \newcommand{\ExtensionTok}[1]{{#1}}
    \newcommand{\PreprocessorTok}[1]{\textcolor[rgb]{0.74,0.48,0.00}{{#1}}}
    \newcommand{\AttributeTok}[1]{\textcolor[rgb]{0.49,0.56,0.16}{{#1}}}
    \newcommand{\InformationTok}[1]{\textcolor[rgb]{0.38,0.63,0.69}{\textbf{\textit{{#1}}}}}
    \newcommand{\WarningTok}[1]{\textcolor[rgb]{0.38,0.63,0.69}{\textbf{\textit{{#1}}}}}
    
    
    % Define a nice break command that doesn't care if a line doesn't already
    % exist.
    \def\br{\hspace*{\fill} \\* }
    % Math Jax compatability definitions
    \def\gt{>}
    \def\lt{<}
    % Document parameters
    \title{Lesson 3}
    
    
    

    % Pygments definitions
    
\makeatletter
\def\PY@reset{\let\PY@it=\relax \let\PY@bf=\relax%
    \let\PY@ul=\relax \let\PY@tc=\relax%
    \let\PY@bc=\relax \let\PY@ff=\relax}
\def\PY@tok#1{\csname PY@tok@#1\endcsname}
\def\PY@toks#1+{\ifx\relax#1\empty\else%
    \PY@tok{#1}\expandafter\PY@toks\fi}
\def\PY@do#1{\PY@bc{\PY@tc{\PY@ul{%
    \PY@it{\PY@bf{\PY@ff{#1}}}}}}}
\def\PY#1#2{\PY@reset\PY@toks#1+\relax+\PY@do{#2}}

\expandafter\def\csname PY@tok@gd\endcsname{\def\PY@tc##1{\textcolor[rgb]{0.63,0.00,0.00}{##1}}}
\expandafter\def\csname PY@tok@gu\endcsname{\let\PY@bf=\textbf\def\PY@tc##1{\textcolor[rgb]{0.50,0.00,0.50}{##1}}}
\expandafter\def\csname PY@tok@gt\endcsname{\def\PY@tc##1{\textcolor[rgb]{0.00,0.27,0.87}{##1}}}
\expandafter\def\csname PY@tok@gs\endcsname{\let\PY@bf=\textbf}
\expandafter\def\csname PY@tok@gr\endcsname{\def\PY@tc##1{\textcolor[rgb]{1.00,0.00,0.00}{##1}}}
\expandafter\def\csname PY@tok@cm\endcsname{\let\PY@it=\textit\def\PY@tc##1{\textcolor[rgb]{0.25,0.50,0.50}{##1}}}
\expandafter\def\csname PY@tok@vg\endcsname{\def\PY@tc##1{\textcolor[rgb]{0.10,0.09,0.49}{##1}}}
\expandafter\def\csname PY@tok@vi\endcsname{\def\PY@tc##1{\textcolor[rgb]{0.10,0.09,0.49}{##1}}}
\expandafter\def\csname PY@tok@vm\endcsname{\def\PY@tc##1{\textcolor[rgb]{0.10,0.09,0.49}{##1}}}
\expandafter\def\csname PY@tok@mh\endcsname{\def\PY@tc##1{\textcolor[rgb]{0.40,0.40,0.40}{##1}}}
\expandafter\def\csname PY@tok@cs\endcsname{\let\PY@it=\textit\def\PY@tc##1{\textcolor[rgb]{0.25,0.50,0.50}{##1}}}
\expandafter\def\csname PY@tok@ge\endcsname{\let\PY@it=\textit}
\expandafter\def\csname PY@tok@vc\endcsname{\def\PY@tc##1{\textcolor[rgb]{0.10,0.09,0.49}{##1}}}
\expandafter\def\csname PY@tok@il\endcsname{\def\PY@tc##1{\textcolor[rgb]{0.40,0.40,0.40}{##1}}}
\expandafter\def\csname PY@tok@go\endcsname{\def\PY@tc##1{\textcolor[rgb]{0.53,0.53,0.53}{##1}}}
\expandafter\def\csname PY@tok@cp\endcsname{\def\PY@tc##1{\textcolor[rgb]{0.74,0.48,0.00}{##1}}}
\expandafter\def\csname PY@tok@gi\endcsname{\def\PY@tc##1{\textcolor[rgb]{0.00,0.63,0.00}{##1}}}
\expandafter\def\csname PY@tok@gh\endcsname{\let\PY@bf=\textbf\def\PY@tc##1{\textcolor[rgb]{0.00,0.00,0.50}{##1}}}
\expandafter\def\csname PY@tok@ni\endcsname{\let\PY@bf=\textbf\def\PY@tc##1{\textcolor[rgb]{0.60,0.60,0.60}{##1}}}
\expandafter\def\csname PY@tok@nl\endcsname{\def\PY@tc##1{\textcolor[rgb]{0.63,0.63,0.00}{##1}}}
\expandafter\def\csname PY@tok@nn\endcsname{\let\PY@bf=\textbf\def\PY@tc##1{\textcolor[rgb]{0.00,0.00,1.00}{##1}}}
\expandafter\def\csname PY@tok@no\endcsname{\def\PY@tc##1{\textcolor[rgb]{0.53,0.00,0.00}{##1}}}
\expandafter\def\csname PY@tok@na\endcsname{\def\PY@tc##1{\textcolor[rgb]{0.49,0.56,0.16}{##1}}}
\expandafter\def\csname PY@tok@nb\endcsname{\def\PY@tc##1{\textcolor[rgb]{0.00,0.50,0.00}{##1}}}
\expandafter\def\csname PY@tok@nc\endcsname{\let\PY@bf=\textbf\def\PY@tc##1{\textcolor[rgb]{0.00,0.00,1.00}{##1}}}
\expandafter\def\csname PY@tok@nd\endcsname{\def\PY@tc##1{\textcolor[rgb]{0.67,0.13,1.00}{##1}}}
\expandafter\def\csname PY@tok@ne\endcsname{\let\PY@bf=\textbf\def\PY@tc##1{\textcolor[rgb]{0.82,0.25,0.23}{##1}}}
\expandafter\def\csname PY@tok@nf\endcsname{\def\PY@tc##1{\textcolor[rgb]{0.00,0.00,1.00}{##1}}}
\expandafter\def\csname PY@tok@si\endcsname{\let\PY@bf=\textbf\def\PY@tc##1{\textcolor[rgb]{0.73,0.40,0.53}{##1}}}
\expandafter\def\csname PY@tok@s2\endcsname{\def\PY@tc##1{\textcolor[rgb]{0.73,0.13,0.13}{##1}}}
\expandafter\def\csname PY@tok@nt\endcsname{\let\PY@bf=\textbf\def\PY@tc##1{\textcolor[rgb]{0.00,0.50,0.00}{##1}}}
\expandafter\def\csname PY@tok@nv\endcsname{\def\PY@tc##1{\textcolor[rgb]{0.10,0.09,0.49}{##1}}}
\expandafter\def\csname PY@tok@s1\endcsname{\def\PY@tc##1{\textcolor[rgb]{0.73,0.13,0.13}{##1}}}
\expandafter\def\csname PY@tok@dl\endcsname{\def\PY@tc##1{\textcolor[rgb]{0.73,0.13,0.13}{##1}}}
\expandafter\def\csname PY@tok@ch\endcsname{\let\PY@it=\textit\def\PY@tc##1{\textcolor[rgb]{0.25,0.50,0.50}{##1}}}
\expandafter\def\csname PY@tok@m\endcsname{\def\PY@tc##1{\textcolor[rgb]{0.40,0.40,0.40}{##1}}}
\expandafter\def\csname PY@tok@gp\endcsname{\let\PY@bf=\textbf\def\PY@tc##1{\textcolor[rgb]{0.00,0.00,0.50}{##1}}}
\expandafter\def\csname PY@tok@sh\endcsname{\def\PY@tc##1{\textcolor[rgb]{0.73,0.13,0.13}{##1}}}
\expandafter\def\csname PY@tok@ow\endcsname{\let\PY@bf=\textbf\def\PY@tc##1{\textcolor[rgb]{0.67,0.13,1.00}{##1}}}
\expandafter\def\csname PY@tok@sx\endcsname{\def\PY@tc##1{\textcolor[rgb]{0.00,0.50,0.00}{##1}}}
\expandafter\def\csname PY@tok@bp\endcsname{\def\PY@tc##1{\textcolor[rgb]{0.00,0.50,0.00}{##1}}}
\expandafter\def\csname PY@tok@c1\endcsname{\let\PY@it=\textit\def\PY@tc##1{\textcolor[rgb]{0.25,0.50,0.50}{##1}}}
\expandafter\def\csname PY@tok@fm\endcsname{\def\PY@tc##1{\textcolor[rgb]{0.00,0.00,1.00}{##1}}}
\expandafter\def\csname PY@tok@o\endcsname{\def\PY@tc##1{\textcolor[rgb]{0.40,0.40,0.40}{##1}}}
\expandafter\def\csname PY@tok@kc\endcsname{\let\PY@bf=\textbf\def\PY@tc##1{\textcolor[rgb]{0.00,0.50,0.00}{##1}}}
\expandafter\def\csname PY@tok@c\endcsname{\let\PY@it=\textit\def\PY@tc##1{\textcolor[rgb]{0.25,0.50,0.50}{##1}}}
\expandafter\def\csname PY@tok@mf\endcsname{\def\PY@tc##1{\textcolor[rgb]{0.40,0.40,0.40}{##1}}}
\expandafter\def\csname PY@tok@err\endcsname{\def\PY@bc##1{\setlength{\fboxsep}{0pt}\fcolorbox[rgb]{1.00,0.00,0.00}{1,1,1}{\strut ##1}}}
\expandafter\def\csname PY@tok@mb\endcsname{\def\PY@tc##1{\textcolor[rgb]{0.40,0.40,0.40}{##1}}}
\expandafter\def\csname PY@tok@ss\endcsname{\def\PY@tc##1{\textcolor[rgb]{0.10,0.09,0.49}{##1}}}
\expandafter\def\csname PY@tok@sr\endcsname{\def\PY@tc##1{\textcolor[rgb]{0.73,0.40,0.53}{##1}}}
\expandafter\def\csname PY@tok@mo\endcsname{\def\PY@tc##1{\textcolor[rgb]{0.40,0.40,0.40}{##1}}}
\expandafter\def\csname PY@tok@kd\endcsname{\let\PY@bf=\textbf\def\PY@tc##1{\textcolor[rgb]{0.00,0.50,0.00}{##1}}}
\expandafter\def\csname PY@tok@mi\endcsname{\def\PY@tc##1{\textcolor[rgb]{0.40,0.40,0.40}{##1}}}
\expandafter\def\csname PY@tok@kn\endcsname{\let\PY@bf=\textbf\def\PY@tc##1{\textcolor[rgb]{0.00,0.50,0.00}{##1}}}
\expandafter\def\csname PY@tok@cpf\endcsname{\let\PY@it=\textit\def\PY@tc##1{\textcolor[rgb]{0.25,0.50,0.50}{##1}}}
\expandafter\def\csname PY@tok@kr\endcsname{\let\PY@bf=\textbf\def\PY@tc##1{\textcolor[rgb]{0.00,0.50,0.00}{##1}}}
\expandafter\def\csname PY@tok@s\endcsname{\def\PY@tc##1{\textcolor[rgb]{0.73,0.13,0.13}{##1}}}
\expandafter\def\csname PY@tok@kp\endcsname{\def\PY@tc##1{\textcolor[rgb]{0.00,0.50,0.00}{##1}}}
\expandafter\def\csname PY@tok@w\endcsname{\def\PY@tc##1{\textcolor[rgb]{0.73,0.73,0.73}{##1}}}
\expandafter\def\csname PY@tok@kt\endcsname{\def\PY@tc##1{\textcolor[rgb]{0.69,0.00,0.25}{##1}}}
\expandafter\def\csname PY@tok@sc\endcsname{\def\PY@tc##1{\textcolor[rgb]{0.73,0.13,0.13}{##1}}}
\expandafter\def\csname PY@tok@sb\endcsname{\def\PY@tc##1{\textcolor[rgb]{0.73,0.13,0.13}{##1}}}
\expandafter\def\csname PY@tok@sa\endcsname{\def\PY@tc##1{\textcolor[rgb]{0.73,0.13,0.13}{##1}}}
\expandafter\def\csname PY@tok@k\endcsname{\let\PY@bf=\textbf\def\PY@tc##1{\textcolor[rgb]{0.00,0.50,0.00}{##1}}}
\expandafter\def\csname PY@tok@se\endcsname{\let\PY@bf=\textbf\def\PY@tc##1{\textcolor[rgb]{0.73,0.40,0.13}{##1}}}
\expandafter\def\csname PY@tok@sd\endcsname{\let\PY@it=\textit\def\PY@tc##1{\textcolor[rgb]{0.73,0.13,0.13}{##1}}}

\def\PYZbs{\char`\\}
\def\PYZus{\char`\_}
\def\PYZob{\char`\{}
\def\PYZcb{\char`\}}
\def\PYZca{\char`\^}
\def\PYZam{\char`\&}
\def\PYZlt{\char`\<}
\def\PYZgt{\char`\>}
\def\PYZsh{\char`\#}
\def\PYZpc{\char`\%}
\def\PYZdl{\char`\$}
\def\PYZhy{\char`\-}
\def\PYZsq{\char`\'}
\def\PYZdq{\char`\"}
\def\PYZti{\char`\~}
% for compatibility with earlier versions
\def\PYZat{@}
\def\PYZlb{[}
\def\PYZrb{]}
\makeatother


    % Exact colors from NB
    \definecolor{incolor}{rgb}{0.0, 0.0, 0.5}
    \definecolor{outcolor}{rgb}{0.545, 0.0, 0.0}



    
    % Prevent overflowing lines due to hard-to-break entities
    \sloppy 
    % Setup hyperref package
    \hypersetup{
      breaklinks=true,  % so long urls are correctly broken across lines
      colorlinks=true,
      urlcolor=urlcolor,
      linkcolor=linkcolor,
      citecolor=citecolor,
      }
    % Slightly bigger margins than the latex defaults
    
    \geometry{verbose,tmargin=1in,bmargin=1in,lmargin=1in,rmargin=1in}
    
    

    \begin{document}
    
    
    \maketitle
    
    

    
    \section{Урок
третий}\label{ux443ux440ux43eux43a-ux442ux440ux435ux442ux438ux439}

    \subsection{Строки и
списки}\label{ux441ux442ux440ux43eux43aux438-ux438-ux441ux43fux438ux441ux43aux438}

\subsubsection{Строки}\label{ux441ux442ux440ux43eux43aux438}

Фрагменты текста в программировании обычно называют строками. Можно
сказать, что строка --- это последовательность символов.

\begin{verbatim}
>>> s1 = "Hello world!" 
>>> s2 = 'This is a string...'
>>> str = s1 + s2
>>> print(str)
Hello world! This is a string...
\end{verbatim}

Если кавычки должны быть частью строки, это можно осуществить следующим
образом:

\begin{verbatim}
>>> s = 'This string has "double quotes"'
>>> print(s)
This string has "double quotes"
>>> print("... and now we use 'single quotations marks'")
... and now we use 'single quotations marks'
\end{verbatim}

Переменными внутри строк можно пользоваться следующим образом:

\begin{verbatim}
>>> score = 103
>>> message = "Your score is %s"
>>> print(message % score)
Your score is 103
\end{verbatim}

В одной строке можно пользоваться более чем одной переменной:

\begin{verbatim}
>>> s1 = 'strawberries'
>>> s2 = 'cherries'
>>> print("Please buy %s and %s!" % (s1, s2))
Please buy strawberries and cherries!
>>> print("Please buy {0} and {1}!".format(s1,s2))
Please buy strawberries and cherries!
\end{verbatim}

\textbf{Умножение строк.} Что получится, если умножить 10 на 5?
Разумеется, 50. А если умножить на 10 букву «a»? Вот что думает об этом
Python:

\begin{verbatim}
>>> print(10 * 'a')
aaaaaaaaaa
>>> print('-'*52)
----------------------------------------------------
\end{verbatim}

Эта особенность может пригодиться для вывода строк с отступом в заданное
число пробелов.

\subsubsection{Списки}\label{ux441ux43fux438ux441ux43aux438}

Списки - это специальный объект языка Python. По сути, список это
упорядоченный массив данных. В списках можно хранить значения разных
типов, например числа:

\begin{verbatim}
>>> some_numbers = [1, 2, 5, 10, 20]
\end{verbatim}

Или строки:

\begin{verbatim}
>>> some_strings = ['the', 'Jack', 'house', 'that', 'built', 'is', 'this']
\end{verbatim}

Или числа и строки вперемежку:

\begin{verbatim}
>>> numbers_and_strings = [1, 2, 3, 'the', 'Jack', 'house', 'that', 'in', 'built', 'is', 'this', 2017]
>>> print(numbers_and_strings)
[1, 2, 3, 'the', 'Jack', 'house', 'that', 'in', 'built', 'is', 'this', 2017]
\end{verbatim}

В списках могут даже храниться другие списки:

\begin{verbatim}
>>> numbers = [1, 2, 3, 4, 5]
>>> strings = ['house', 'building', 'shed']
>>> mylist = [numbers, strings]
>>> print(mylist)
[[1, 2, 3, 4, 5], ['house', 'building', 'shed']]
\end{verbatim}

Мы создали три переменные: numbers с пятью цифрами, strings с тремя
строками и mylist, где хранятся списки numbers и strings. Причем в
третьем списке (mylist) только два элемента, ведь он содержит два других
списка, а не их отдельные элементы.

\textbf{Добавление элементов в список}

Для добавления в список новых элементов служит функция append. Функция -
это фрагмент кода, который выполняет какую-то задачу. В данном случае
append добавляет элемент к концу списка.

\begin{verbatim}
>>> animals = ['lions', 'foxes', 'bears', 'wolves', 'turtles']
>>> animals.append('whales')
>>> animals.append('tigers')
>>> print(animals)
['lions', 'foxes', 'bears', 'wolves', 'turtles', 'whales', 'tigers']
\end{verbatim}

Мы можем получить доступ к элементам списка следующим образом:

\begin{verbatim}
>>> print(animals[2])
bears
>>> print(animals[3:5])
['wolves', 'turtles']
>>> print(animals[-1])
tigers
>>> print(animals[-2])
whales
>>> print(animals)
['lions', 'foxes', 'bears', 'wolves', 'turtles', 'whales', 'tigers']
>>> print(animals[0:-1])
['lions', 'foxes', 'bears', 'wolves', 'turtles', 'whales']
>>> print(animals[0:-2])
['lions', 'foxes', 'bears', 'wolves', 'turtles']
\end{verbatim}

Обратите внимание, что первый элемент в списке имеет нулевой индекс:

\begin{verbatim}
>>> print(animals[0])
'lions'
\end{verbatim}

\textbf{Удаление элементов из списка}

\begin{verbatim}
>>> del animals[5]
>>> print(animals)
['lions', 'foxes', 'bears', 'wolves', 'turtles', 'tigers']
\end{verbatim}

\textbf{Списковая арифметика}

Списки можно объединять, складывая их так же, как числа, с помощью знака
«плюс» (\texttt{+}). Например, у нас есть два списка: list1, в котором
хранятся числа от 1 до 4, и list2, где хранится несколько слов. Тогда мы
можем сложить их, воспользовавшись командой print и знаком \texttt{+}.
Вот так:

\begin{verbatim}
>>> strings = ['house', 'building', 'shed']
>>> numbers = [1, 2, 3, 4, 5]
>>> print(strings + numbers)
['house', 'building', 'shed', 1, 2, 3, 4, 5]
\end{verbatim}

Результат сложения двух списков можно поместить в другую переменную:

\begin{verbatim}
>>> list1 = [1, 2, 3, 4]
>>> list2 = ['I', 'like', ice cream']
>>> list3 = list1 + list2
>>> print(list3)
[1, 2, 3, 4, 'I', 'like', 'ice cream']
\end{verbatim}

Также список можно умножать на число с помощью оператора (*). Например,
умножим list1 на 5:

\begin{verbatim}
>>> list1 = [1, 2]
>>> print(list1 * 5)
[1, 2, 1, 2, 1, 2, 1, 2, 1, 2]
\end{verbatim}

Фактически это означает «повторить list1 пять раз», поэтому в итоге
получается 1, 2, 1, 2, 1, 2, 1, 2, 1, 2. Но обратите внимание ---
деление и вычитание со списками не работают. При попытке сделать это вы
получите ошибку!

\subsection{Kортежи и
словари}\label{kux43eux440ux442ux435ux436ux438-ux438-ux441ux43bux43eux432ux430ux440ux438}

\subsubsection{Кортежи}\label{ux43aux43eux440ux442ux435ux436ux438}

Кортеж (tuple) похож на список, элементы которого записаны в круглых
скобках, как в этом примере:

\begin{verbatim}
>>> fibs = (0, 1, 1, 2, 3, 5, 8)
>>> print(fibs[3])
2
\end{verbatim}

Мы определили переменную \texttt{fibs} как набор чисел 0, 1, 1, 2 и 3. И
точно так же, как со списками, мы напечатали элемент с индексом 3 с
помощью команды \texttt{print(fibs{[}3{]})}. Главное отличие кортежа от
списка в том, что кортеж невозможно изменить после его создания.
Например, если мы попытаемся поменять первое значение кортежа
\texttt{fibs} на число 4 (таким же образом, каким меняли значения в
списке \texttt{wizard\_list}), мы получим сообщение об ошибке:

\begin{verbatim}
>>> fibs[0] = 4
Traceback (most recent call last):
File "<pyshell>", line 1, in <module>
fibs[0] = 4
TypeError: 'tuple' object does not support item assignment
\end{verbatim}

Но в чем смысл использования кортежей, если есть списки? Главная причина
такова: порой удобно использовать набор значений, который никогда не
меняется. Создав кортеж с двумя элементами, можно не сомневаться, что в
нем и дальше будут только эти два элемента.

\subsubsection{Словари в Python --- не для поиска
слов}\label{ux441ux43bux43eux432ux430ux440ux438-ux432-python-ux43dux435-ux434ux43bux44f-ux43fux43eux438ux441ux43aux430-ux441ux43bux43eux432}

Словарями в Python называются наборы значений аналогично спискам и
кортежам. Отличие состоит в том, что каждому элементу словаря
соответствуют ключ и связанное с ним значение. Например, у нас есть
перечень людей и их любимых видов спорта. Можно поместить эту информацию
в список, где следом за именем человека указан вид спорта. Вот так:

\begin{verbatim}
>>> favourite_sports = {'Andrew': 'ping pong', 'Michael': 'football', 'Vera': 'netball', 'John': 'chess'}
>>> print(favourite_sports)
{'Vera': 'netball', 'John': 'chess', 'Michael': 'football', 'Andrew': 'ping pong'}
\end{verbatim}

Для разделения каждой пары «ключ--значение» мы использовали двоеточие,
записав при этом ключ и значение в одинарных кавычках. Также обратите
внимание, что элементы словаря заключены в фигурные (а не круглые или
квадратные) скобки. В результате получается словарь, где каждому ключу
соответствует определенное значение.

Теперь, чтобы узнать любимый вид спорта человека записанного под именем
John, нужно обратиться к словарю favorite\_sports, использовав его имя в
качестве ключа:

\begin{verbatim}
>>> print(favorite_sports['John'])
chess
\end{verbatim}

Чтобы удалить значение из словаря, тоже используется ключ. Например,
удалим пару с ключом 'Andrew':

\begin{verbatim}
>>> del favourite_sports['Andrew']
>>> print favourite_sports
{'Vera': 'netball', 'John': 'chess', 'Michael': 'football'}
\end{verbatim}

Ключ нужен и для замены значения в словаре:

\begin{verbatim}
>>> favourite_sports['Vera'] = 'chess'
>>> favourite_sports
{'Vera': 'chess', 'John': 'chess', 'Michael': 'football'}
\end{verbatim}

Как видите, работа со словарями напоминает работу со списками и
кортежами, однако объединять словари с помощью оператора "+" нельзя.
Попытавшись это сделать, вы получите сообщение об ошибке - Python
отказывается объединять словари, потому что не знает, как это делать.

Итак, словари в языке Python это наборы пар «ключ--значение». Мы знаем
как получить значение определенного ключа. А как получить все ключи с
заданным значением? Например, как найти всех людей из словаря
\texttt{favourite\_sports}у которых любимый вид спорта chess? На этот
вопрос мы сумеем ответить только после того как изучим другие
конструкции языка Python (например циклы, генераторы и итераторы).

\subsubsection{Упражнения}\label{ux443ux43fux440ux430ux436ux43dux435ux43dux438ux44f}

\begin{enumerate}
\def\labelenumi{\arabic{enumi}.}
\item
  \textbf{Любимые вещи}. Создайте список своих любимых развлечений и
  сохраните его в переменной games. Теперь создайте список любимых
  лакомств, сохранив его в переменной foods. Объедините два этих списка,
  сохранив результат в переменной favorites, и напечатайте значение этой
  переменной.
\item
  \textbf{Работа со словарем}. Предположим, что
  \texttt{favourite\_sports\ =\ \{"Andrew":\ "ping\ pong",\ "Michael":\ "football",\ "Vera":\ "netball",\ "John":\ "chess"\}}.
  Предположим, что вы не знаете значений имеющихся записей, но знаете,
  что \texttt{Vera} и \texttt{Andrew} это существующие ключи.
  Воспользуйтесь этим словарем и напечатайте следующее предложение:
  \texttt{Vera\ likes\ playing\ ...,\ but\ Andrew\ prefers\ ...}, где
  вместо троеточий должны следовать виды спорта, которые предпочитают
  Vera и Andrew.
\item
  \textbf{Словари и комбинированные ключи.} Создайте словарь, в котором
  ключи - имена людей ("Andrew", "Michael", "Vera", "John"), а
  значениями являются набор информации о любимых развлечениях и любимых
  лакомствах (это можно сделать воспользовавшись кортежами или
  списками). Напишите набор команд, которые распечатывают информацию о
  человеке с именем \texttt{name}.
\end{enumerate}


    % Add a bibliography block to the postdoc
    
    
    
    \end{document}
